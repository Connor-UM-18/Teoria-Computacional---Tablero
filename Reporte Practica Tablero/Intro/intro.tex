\chapter{Introducción}
En esta práctica de Teoría de la Computación vamos a trabajar en el diseño e implementación de un programa de ajedrez que permita realizar movimientos ortogonales y diagonales en un tablero de 4x4, considerando la opción de tener una o dos piezas en el tablero. Además de cumplir con las reglas y movimientos establecidos en las láminas del curso de Stanford, el programa debe contar con las siguientes características:\newline

Se siguieron las siguientes instrucciones especificadas por el docente, que son las siguientes:\newline
\newline
\begin{enumerate}
    \item Modo automático y modo manual: El programa debe poder ejecutarse tanto en modo automático como en modo manual. En el modo manual, el usuario tiene la opción de introducir la cadena de movimientos o generarla aleatoriamente.
    \newline
    \item Elección de la cantidad de piezas: El programa debe ser capaz de funcionar con una o dos piezas en el tablero. Si se selecciona la opción de dos piezas, la segunda pieza debe iniciar en el estado 4 y su estado final debe ser el estado 13.
    \newline
    \item Elección aleatoria del jugador inicial: Al iniciar el juego, el programa debe decidir de manera aleatoria quién comienza a jugar.
    \newline
    \item Generación de archivos de movimientos: Una vez que se define la cadena de movimientos para una o dos piezas, el programa debe generar archivos que contengan todos los movimientos posibles por pieza, así como otro archivo con los movimientos ganadores por pieza. Estos archivos serán utilizados para reconfigurar las rutas en el juego.
    \newline
    \item Reconfiguración de rutas y espera en caso de no poder avanzar: Si se reconfigura una ruta y aun así no es posible avanzar, se debe esperar una iteración antes de intentar otra reconfiguración.
    \newline
    \item Graficación del tablero y visualización de movimientos: El programa debe permitir graficar el tablero de ajedrez y mostrar los movimientos realizados por una o dos piezas. Esto facilita la visualización interactiva de los movimientos en el tablero.
    \newline
    \item Restricción de movimientos automáticos: En el modo automático, las cadenas generadas no deben tener más de 10 movimientos para poder ser utilizadas en la animación.
    \newline
    \item Límite máximo de movimientos: El programa debe imponer un límite de 100 símbolos como máximo para la cantidad de movimientos permitidos en una cadena.
    \newline
    \item Generación de árbol de movimientos: Se debe generar un archivo de salida en formato de imagen que represente el árbol de movimientos evaluados en cada corrida del programa.
    \newline
    \item Inclusión del código fuente en el reporte: El reporte final debe incluir el código fuente completo del programa desarrollado.
    \newline
    
\end{enumerate}
    
Además, se deben cumplir las características y requerimientos mencionados anteriormente, incluyendo la generación de archivos de movimientos, la graficación del tablero, la restricción de movimientos automáticos y la generación de árboles de movimientos. El reporte final debe incluir el código fuente completo del programa.\newline